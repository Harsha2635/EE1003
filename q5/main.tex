\let\negmedspace\undefined
\let\negthickspace\undefined
\documentclass[journal]{IEEEtran}
\usepackage[a5paper, margin=10mm, onecolumn]{geometry}
%\usepackage{lmodern} % Ensure lmodern is loaded for pdflatex
\usepackage{tfrupee} % Include tfrupee package

\setlength{\headheight}{1cm} % Set the height of the header box
\setlength{\headsep}{0mm}     % Set the distance between the header box and the top of the text

\usepackage{gvv-book}
\usepackage{gvv}
\usepackage{cite}
\usepackage{amsmath,amssymb,amsfonts,amsthm}
\usepackage{algorithmic}
\usepackage{graphicx}
\usepackage{textcomp}
\usepackage{xcolor}
\usepackage{txfonts}
\usepackage{listings}
\usepackage{enumitem}
\usepackage{mathtools}
\usepackage{gensymb}
\usepackage{comment}
\usepackage[breaklinks=true]{hyperref}
\usepackage{tkz-euclide} 
\usepackage{listings}
% \usepackage{gvv}                                        
\def\inputGnumericTable{}                                 
\usepackage[latin1]{inputenc}                                
\usepackage{color}                                            
\usepackage{array}                                            
\usepackage{longtable}                                       
\usepackage{calc}                                             
\usepackage{multirow}                                         
\usepackage{hhline}                                           
\usepackage{ifthen}                                           
\usepackage{lscape}
\usepackage{circuitikz}
\tikzstyle{block} = [rectangle, draw, fill=blue!20, 
    text width=4em, text centered, rounded corners, minimum height=3em]
\tikzstyle{sum} = [draw, fill=blue!10, circle, minimum size=1cm, node distance=1.5cm]
\tikzstyle{input} = [coordinate]
\tikzstyle{output} = [coordinate]


\begin{document}

\bibliographystyle{IEEEtran}
\vspace{3cm}

\title{8.6-6.5-1.1}
\author{EE24BTECH11063 - Y. Harsha Vardhan Reddy}
 \maketitle
% \newpage
% \bigskip
{\let\newpage\relax\maketitle}

\renewcommand{\thefigure}{\theenumi}
\renewcommand{\thetable}{\theenumi}
\setlength{\intextsep}{10pt} % Space between text and floats


\numberwithin{equation}{enumi}
\numberwithin{figure}{enumi}
\renewcommand{\thetable}{\theenumi}

\textbf{Question}:\\
Find the minimum value of the function\\
$$f(x)=\brak{2x-1}^2+3$$\\
\solution \\
\\
\textbf{Theoritical solution:}\\
Given,\\
\begin{align}
    \frac{dy}{dx}=4\brak{2x-1}=0\\
    \implies x=\frac{1}{2}\\
    \frac{d^2y}{dx^2}=8\\
\end{align}
Since, $\frac{d^2y}{dx^2}>0$, at $x=\frac{1}{2}$ there exists minimum  \\
Therefore, $f\brak{\frac{1}{2}}=3$ is the minimum value of the function\\
\textbf{Computational Solution Using Gradient Descent} \\
To verify the analytical results, we use gradient descent to find the local minimum \\
Gradient Descent for local minimum : \\ 
 - Start with $x_0 = 4$ \\
 - Update $x$ iteratively using 
\begin{align}
    x_{n+1} = x_n - \eta \cdot f'(x_n)
\end{align}
where :
\begin{align}
    \eta = 0.1 
\end{align}
\begin{align}
    f'(x) = 4(2x-1) 
\end{align}
\begin{align}
    x_{n+1} = x_n - \eta \cdot (4(2x_n-1))
\end{align}
\textbf{Computational Results} \\
 - Local minimum 
 \begin{align}
     x \approx 0.5,\text{ } g(x) \approx 3.000
 \end{align} 
 \textbf{Computational Solution Using Quadratic programming problem} \\

We aim to find the minimum value of the quadratic function:
\[
f(x) = (2x - 1)^2 + 3
\]
Expanding the terms, we get:
\[
f(x) = 4x^2 - 4x + 1 + 3 = 4x^2 - 4x + 4
\]

\textbf{Formulating the Problem as Quadratic Programming}
The general form of a quadratic programming problem is:
\[
\text{Minimize } \frac{1}{2}x^\top Q x + c^\top x
\]
where:
\begin{itemize}
    \item \( Q \) is the coefficient matrix for the quadratic term,
    \item \( c \) is the coefficient vector for the linear term.
\end{itemize}

For the given function:
\[
f(x) = 4x^2 - 4x + 4
\]
we identify:
\[
Q = 4, \quad c = -4
\]
The constant term \( +4 \) does not affect the minimization process but will be added back to compute the final minimum value.

\textbf{Steps to Solve Using cvxpy}\\
We use the Python library \texttt{cvxpy} to solve this quadratic programming problem. The steps are as follows:
\begin{enumerate}
    \item Define the variable \( x \) to be optimized.
    \item Write the objective function \( \frac{1}{2} Q x^2 + c x \) in terms of \( x \).
    \item Solve the problem using \texttt{cvxpy}.
    \item Add the constant term \( +4 \) to the resulting minimum value of the objective function.
\end{enumerate}


\textbf{Computational solution}\\
From the code, the solution is:
\begin{itemize}
    \item The value of \( x \) at the minimum is:
    \[
    x = \frac{-c}{2Q} = \frac{-(-4)}{2(4)} = \frac{4}{8} = 0.5
    \]
    \item The minimum value of the function is:
    \[
    f(0.5) = 4(0.5)^2 - 4(0.5) + 4 = 1 - 2 + 4 = 3
    \]
\end{itemize}

Thus, the minimum value of the function is:
\[
\boxed{3}
\]
and it occurs at:
\[
\boxed{x = 0.5}
\] 
 \begin{figure}[ht!]
   \centering
   \includegraphics[width=\columnwidth]{figs/Figure_1.png}
\end{figure}
\end{document}

