\let\negmedspace\undefined
\let\negthickspace\undefined
\documentclass[journal]{IEEEtran}
\usepackage[a5paper, margin=10mm, onecolumn]{geometry}
%\usepackage{lmodern} % Ensure lmodern is loaded for pdflatex
\usepackage{tfrupee} % Include tfrupee package

\setlength{\headheight}{1cm} % Set the height of the header box
\setlength{\headsep}{0mm}     % Set the distance between the header box and the top of the text

\usepackage{gvv-book}
\usepackage{gvv}
\usepackage{cite}
\usepackage{amsmath,amssymb,amsfonts,amsthm}
\usepackage{algorithmic}
\usepackage{graphicx}
\usepackage{textcomp}
\usepackage{xcolor}
\usepackage{txfonts}
\usepackage{listings}
\usepackage{enumitem}
\usepackage{mathtools}
\usepackage{gensymb}
\usepackage{comment}
\usepackage[breaklinks=true]{hyperref}
\usepackage{tkz-euclide} 
\usepackage{listings}
% \usepackage{gvv}                                        
\def\inputGnumericTable{}                                 
\usepackage[latin1]{inputenc}                                
\usepackage{color}                                            
\usepackage{array}                                            
\usepackage{longtable}                                       
\usepackage{calc}                                             
\usepackage{multirow}                                         
\usepackage{hhline}                                           
\usepackage{ifthen}                                           
\usepackage{lscape}
\usepackage{circuitikz}
\tikzstyle{block} = [rectangle, draw, fill=blue!20, 
    text width=4em, text centered, rounded corners, minimum height=3em]
\tikzstyle{sum} = [draw, fill=blue!10, circle, minimum size=1cm, node distance=1.5cm]
\tikzstyle{input} = [coordinate]
\tikzstyle{output} = [coordinate]


\begin{document}

\bibliographystyle{IEEEtran}
\vspace{3cm}

\title{9-9.5-11}
\author{EE24BTECH11063 - Y. Harsha Vardhan Reddy}
 \maketitle
% \newpage
% \bigskip
{\let\newpage\relax\maketitle}

\renewcommand{\thefigure}{\theenumi}
\renewcommand{\thetable}{\theenumi}
\setlength{\intextsep}{10pt} % Space between text and floats


\numberwithin{equation}{enumi}
\numberwithin{figure}{enumi}
\renewcommand{\thetable}{\theenumi}

\textbf{Question}:\\
For the following differential equation, find the particular solution satisfying the given condition:\\
$(x + y)\;dy + (x - y)\;dx = 0;\;y=1 \text{ when } x=1$\\
\solution \\
First let us solve the given differential equation theoritically and then do it computationally and verify if they are equal \\
\begin{align}
    {\brak{x + y}\;dy + \brak{x - y}\;dx = 0}
\end{align}
or \begin{align}
    \frac{dy}{dx}=\frac{y-x}{y+x}
\end{align}
Let \begin{align}
    F\brak{x,y}=\frac{y-x}{y+x}
\end{align}
then \begin{align}
    F\brak{\alpha x,\alpha y}={\alpha}^0 \times \frac{y-x}{y+x}
\end{align}
Therefore, this is a homogeneous equation in x and y\\
Let \begin{align}
    y = v x 
\end{align}
or \begin{align}
    v + x \frac{dv}{dx} = \frac{v-1}{v+1}
\end{align}
or \begin{align}
    x\frac{dv}{dx}=\frac{v-1}{v+1}-v
\end{align}
or \begin{align}
    x\frac{dv}{dx}=-\frac{v^2+1}{v+1}
\end{align}
or \begin{align}
    \frac{v+1}{v^2+1}\cdot dv = -\frac{dx}{x}
\end{align}
Integrating on Both sides, \begin{align}
   \int \frac{v}{v^2+1}\cdot dv + \int\frac{1}{v^2+1}\cdot dv = -\int \frac{dx}{x}
\end{align}
or \begin{align}
    \frac{1}{2}\ln(v^2+1) + \tan^{-1}{\brak{v}} = - \ln{x} + c
\end{align}
By sustituting $v=\frac{y}{x}$ ,
\begin{align}
    \frac{1}{2}\ln \brak{\frac{y^2}{x^2}+1} + \tan^{-1}{\brak{\frac{y}{x}}} = - \ln{x} + c
    \label{0.12}
\end{align}
where, c is the constant of integration
By substituting $x=1 \text{ and } y=1$ in \ref{0.12},
\begin{align}
    c = \frac{1}{2}\ln{2} + \frac{\pi}{4}
\end{align}
Final equation,
\begin{align}
    \frac{1}{2}\ln \brak{\frac{y^2}{x^2}+1} + \tan^{-1}{\brak{\frac{y}{x}}} = - \ln{x} + \frac{1}{2}\ln{2} + \frac{\pi}{4}
    \label{0.14}
\end{align}
Now let us verify this computationally
From definition of $\frac{dy}{dx}$,
\begin{align}
    y_{n+1}=y_{n}+\frac{dy}{dx}\cdot h
    \label{0.15}
\end{align}
(where h is small number tending to zero)
From the differential equation given,
\begin{align}
    \frac{dy}{dx}=\frac{y_n-x_n}{y_n+x_n}
    \label{0.16}
\end{align}
By substituting \ref{0.16} in \ref{0.15},
\begin{align}
    y_{n+1}=y_{n}+\brak{\frac{y_n-x_n}{y_n+x_n}}\cdot h
\end{align}
By taking $x_1=1 \text{ and } y_1=1$  and $h=0.04$ going till $x=3$ by iterating through the loop and finding $y_2,y_3,\cdots , y_{500}$ and plotting the graph the implicit function \ref{0.14} we can verify if the function we got by solving the differential equation mathematically.
\end{document}



